\PassOptionsToPackage{unicode=true}{hyperref} % options for packages loaded elsewhere
\PassOptionsToPackage{hyphens}{url}
%
\documentclass[
]{article}
\usepackage{lmodern}
\usepackage{amssymb,amsmath}
\usepackage{ifxetex,ifluatex}
\ifnum 0\ifxetex 1\fi\ifluatex 1\fi=0 % if pdftex
  \usepackage[T1]{fontenc}
  \usepackage[utf8]{inputenc}
  \usepackage{textcomp} % provides euro and other symbols
\else % if luatex or xelatex
  \usepackage{unicode-math}
  \defaultfontfeatures{Scale=MatchLowercase}
  \defaultfontfeatures[\rmfamily]{Ligatures=TeX,Scale=1}
\fi
% use upquote if available, for straight quotes in verbatim environments
\IfFileExists{upquote.sty}{\usepackage{upquote}}{}
\IfFileExists{microtype.sty}{% use microtype if available
  \usepackage[]{microtype}
  \UseMicrotypeSet[protrusion]{basicmath} % disable protrusion for tt fonts
}{}
\makeatletter
\@ifundefined{KOMAClassName}{% if non-KOMA class
  \IfFileExists{parskip.sty}{%
    \usepackage{parskip}
  }{% else
    \setlength{\parindent}{0pt}
    \setlength{\parskip}{6pt plus 2pt minus 1pt}}
}{% if KOMA class
  \KOMAoptions{parskip=half}}
\makeatother
\usepackage{xcolor}
\IfFileExists{xurl.sty}{\usepackage{xurl}}{} % add URL line breaks if available
\IfFileExists{bookmark.sty}{\usepackage{bookmark}}{\usepackage{hyperref}}
\hypersetup{
  pdftitle={Data Basics},
  pdfauthor={Lt Col Ken Horton; Professor Bradley Warner},
  pdfborder={0 0 0},
  breaklinks=true}
\urlstyle{same}  % don't use monospace font for urls
\usepackage[margin=1in]{geometry}
\usepackage{color}
\usepackage{fancyvrb}
\newcommand{\VerbBar}{|}
\newcommand{\VERB}{\Verb[commandchars=\\\{\}]}
\DefineVerbatimEnvironment{Highlighting}{Verbatim}{commandchars=\\\{\}}
% Add ',fontsize=\small' for more characters per line
\usepackage{framed}
\definecolor{shadecolor}{RGB}{248,248,248}
\newenvironment{Shaded}{\begin{snugshade}}{\end{snugshade}}
\newcommand{\AlertTok}[1]{\textcolor[rgb]{0.94,0.16,0.16}{#1}}
\newcommand{\AnnotationTok}[1]{\textcolor[rgb]{0.56,0.35,0.01}{\textbf{\textit{#1}}}}
\newcommand{\AttributeTok}[1]{\textcolor[rgb]{0.77,0.63,0.00}{#1}}
\newcommand{\BaseNTok}[1]{\textcolor[rgb]{0.00,0.00,0.81}{#1}}
\newcommand{\BuiltInTok}[1]{#1}
\newcommand{\CharTok}[1]{\textcolor[rgb]{0.31,0.60,0.02}{#1}}
\newcommand{\CommentTok}[1]{\textcolor[rgb]{0.56,0.35,0.01}{\textit{#1}}}
\newcommand{\CommentVarTok}[1]{\textcolor[rgb]{0.56,0.35,0.01}{\textbf{\textit{#1}}}}
\newcommand{\ConstantTok}[1]{\textcolor[rgb]{0.00,0.00,0.00}{#1}}
\newcommand{\ControlFlowTok}[1]{\textcolor[rgb]{0.13,0.29,0.53}{\textbf{#1}}}
\newcommand{\DataTypeTok}[1]{\textcolor[rgb]{0.13,0.29,0.53}{#1}}
\newcommand{\DecValTok}[1]{\textcolor[rgb]{0.00,0.00,0.81}{#1}}
\newcommand{\DocumentationTok}[1]{\textcolor[rgb]{0.56,0.35,0.01}{\textbf{\textit{#1}}}}
\newcommand{\ErrorTok}[1]{\textcolor[rgb]{0.64,0.00,0.00}{\textbf{#1}}}
\newcommand{\ExtensionTok}[1]{#1}
\newcommand{\FloatTok}[1]{\textcolor[rgb]{0.00,0.00,0.81}{#1}}
\newcommand{\FunctionTok}[1]{\textcolor[rgb]{0.00,0.00,0.00}{#1}}
\newcommand{\ImportTok}[1]{#1}
\newcommand{\InformationTok}[1]{\textcolor[rgb]{0.56,0.35,0.01}{\textbf{\textit{#1}}}}
\newcommand{\KeywordTok}[1]{\textcolor[rgb]{0.13,0.29,0.53}{\textbf{#1}}}
\newcommand{\NormalTok}[1]{#1}
\newcommand{\OperatorTok}[1]{\textcolor[rgb]{0.81,0.36,0.00}{\textbf{#1}}}
\newcommand{\OtherTok}[1]{\textcolor[rgb]{0.56,0.35,0.01}{#1}}
\newcommand{\PreprocessorTok}[1]{\textcolor[rgb]{0.56,0.35,0.01}{\textit{#1}}}
\newcommand{\RegionMarkerTok}[1]{#1}
\newcommand{\SpecialCharTok}[1]{\textcolor[rgb]{0.00,0.00,0.00}{#1}}
\newcommand{\SpecialStringTok}[1]{\textcolor[rgb]{0.31,0.60,0.02}{#1}}
\newcommand{\StringTok}[1]{\textcolor[rgb]{0.31,0.60,0.02}{#1}}
\newcommand{\VariableTok}[1]{\textcolor[rgb]{0.00,0.00,0.00}{#1}}
\newcommand{\VerbatimStringTok}[1]{\textcolor[rgb]{0.31,0.60,0.02}{#1}}
\newcommand{\WarningTok}[1]{\textcolor[rgb]{0.56,0.35,0.01}{\textbf{\textit{#1}}}}
\usepackage{longtable,booktabs}
% Allow footnotes in longtable head/foot
\IfFileExists{footnotehyper.sty}{\usepackage{footnotehyper}}{\usepackage{footnote}}
\makesavenoteenv{longtable}
\usepackage{graphicx,grffile}
\makeatletter
\def\maxwidth{\ifdim\Gin@nat@width>\linewidth\linewidth\else\Gin@nat@width\fi}
\def\maxheight{\ifdim\Gin@nat@height>\textheight\textheight\else\Gin@nat@height\fi}
\makeatother
% Scale images if necessary, so that they will not overflow the page
% margins by default, and it is still possible to overwrite the defaults
% using explicit options in \includegraphics[width, height, ...]{}
\setkeys{Gin}{width=\maxwidth,height=\maxheight,keepaspectratio}
\setlength{\emergencystretch}{3em}  % prevent overfull lines
\providecommand{\tightlist}{%
  \setlength{\itemsep}{0pt}\setlength{\parskip}{0pt}}
\setcounter{secnumdepth}{-2}
% Redefines (sub)paragraphs to behave more like sections
\ifx\paragraph\undefined\else
  \let\oldparagraph\paragraph
  \renewcommand{\paragraph}[1]{\oldparagraph{#1}\mbox{}}
\fi
\ifx\subparagraph\undefined\else
  \let\oldsubparagraph\subparagraph
  \renewcommand{\subparagraph}[1]{\oldsubparagraph{#1}\mbox{}}
\fi

% set default figure placement to htbp
\makeatletter
\def\fps@figure{htbp}
\makeatother


\title{Data Basics}
\author{Lt Col Ken Horton \and Professor Bradley Warner}
\date{13 May, 2020}

\begin{document}
\maketitle

\newcommand{\E}{\mbox{E}}
\newcommand{\Var}{\mbox{Var}}
\newcommand{\Cov}{\mbox{Cov}}
\newcommand{\Prob}{\mbox{P}}
\newcommand*\diff{\mathop{}\!\mathrm{d}}

\hypertarget{objectives}{%
\section{Objectives}\label{objectives}}

\begin{enumerate}
\def\labelenumi{\arabic{enumi})}
\item
  Define and use properly in context all new terminology to include but
  not limited to case, observational unit, variables, data frame,
  associated variables, independent, and discrete and continuous
  variables.
\item
  Identify and define the different types of variables.
\item
  From reading a study, explain the research question.
\item
  Create a scatterplot in \texttt{R} and determine the association of
  two numerical variables from the plot.
\end{enumerate}

\hypertarget{data-basics}{%
\subsection{Data basics}\label{data-basics}}

Effective presentation and description of data is a first step in most
analyses. This lesson introduces one structure for organizing data as
well as some terminology that will be used throughout this book.

\hypertarget{observations-variables-and-data-matrices}{%
\subsubsection{Observations, variables, and data
matrices}\label{observations-variables-and-data-matrices}}

For reference we will be using a data set concerning 50 emails received
in 2012. These observations will be referred to as the \texttt{email50}
data set, and they are a random sample from a larger data set. This data
is in the \texttt{openintro} package so let's load our packages.

\begin{Shaded}
\begin{Highlighting}[]
\KeywordTok{library}\NormalTok{(tidyverse)}
\KeywordTok{library}\NormalTok{(mosaic)}
\KeywordTok{library}\NormalTok{(openintro)}
\KeywordTok{library}\NormalTok{(knitr)}
\end{Highlighting}
\end{Shaded}

INCOMPLETE THOUGHT: The table below shows 5 rows of the \texttt{email50}
data set concerning\ldots{}.? Should we mention that this only contains
a few observations AND only a few characteristics?

Each row in the table represents a single email or
\textbf{case}.\footnote{A case is also sometimes called a \textbf{unit
  of observation} or an \textbf{observational unit}} The columns
represent characteristics, called \textbf{variables}, for each of the
emails. For example, the first row represents email 1, which is not
spam, contains 21,705 characters, 551 line breaks, is written in HTML
format, and contains only small numbers.

\begin{longtable}[]{@{}lrrrrl@{}}
\toprule
& spam & num\_char & line\_breaks & format & number\tabularnewline
\midrule
\endhead
1 & 0 & 21.705 & 551 & 1 & small\tabularnewline
2 & 0 & 7.011 & 183 & 1 & big\tabularnewline
3 & 1 & 0.631 & 28 & 0 & none\tabularnewline
50 & 0 & 15.829 & 242 & 1 & small\tabularnewline
\bottomrule
\end{longtable}

Let's look at the first 10 rows of data from \texttt{email50}. Remember
to ask the two questions:

\emph{What do we want \texttt{R} to do?} and \emph{What must we give
\texttt{R} for it to do this?}

We want the first 10 rows so we use \texttt{head} and \texttt{R} needs
the data object and the number of rows. The data object is called
\texttt{email50} and is accessible once the \texttt{openintro} package
is loaded.

\begin{Shaded}
\begin{Highlighting}[]
\KeywordTok{head}\NormalTok{(email50,}\DataTypeTok{n=}\DecValTok{10}\NormalTok{)}
\end{Highlighting}
\end{Shaded}

\begin{verbatim}
##    spam to_multiple from cc sent_email                time image attach dollar
## 1     0           0    1  0          1 2012-01-04 06:19:16     0      0      0
## 2     0           0    1  0          0 2012-02-16 13:10:06     0      0      0
## 3     1           0    1  4          0 2012-01-04 08:36:23     0      2      0
## 4     0           0    1  0          0 2012-01-04 10:49:52     0      0      0
## 5     0           0    1  0          0 2012-01-27 02:34:45     0      0      9
## 6     0           0    1  0          0 2012-01-17 10:31:57     0      0      0
## 7     0           0    1  0          0 2012-03-17 22:18:55     0      0      0
## 8     0           0    1  0          1 2012-03-31 07:58:56     0      0      0
## 9     0           0    1  1          1 2012-01-10 18:57:54     0      0      0
## 10    0           0    1  0          0 2012-01-07 12:29:16     0      0     23
##    winner inherit viagra password num_char line_breaks format re_subj
## 1      no       0      0        0   21.705         551      1       1
## 2      no       0      0        0    7.011         183      1       0
## 3      no       0      0        0    0.631          28      0       0
## 4      no       0      0        0    2.454          61      0       0
## 5      no       0      0        1   41.623        1088      1       0
## 6      no       0      0        0    0.057           5      0       0
## 7      no       0      0        0    0.809          17      0       0
## 8      no       0      0        0    5.229          88      1       1
## 9      no       0      0        0    9.277         242      1       1
## 10     no       0      0        0   17.170         578      1       0
##    exclaim_subj urgent_subj exclaim_mess number
## 1             0           0            8  small
## 2             0           0            1    big
## 3             0           0            2   none
## 4             0           0            1  small
## 5             0           0           43  small
## 6             0           0            0  small
## 7             0           0            0  small
## 8             0           0            2  small
## 9             1           0           22  small
## 10            0           0            3  small
\end{verbatim}

In practice, it is especially important to ask clarifying questions to
ensure important aspects of the data are understood. For instance, it is
always important to be sure we know what each variable means and the
units of measurement. Descriptions of all variables in the
\texttt{email50} data set are given in its documentation which can be
accessed in \texttt{R} by using the \texttt{?} command:

\begin{verbatim}
?email50
\end{verbatim}

(Note that not all data sets will have associated documentation; the
authors of \texttt{openintro} package included this documentation with
the \texttt{email50} dataset contained in the package.)

The data in \texttt{email50} represent a \textbf{data matrix} or in
\texttt{R} terminology \textbf{data frame}, which is a common way to
organize data. Each row of a data matrix corresponds to a unique case,
and each column corresponds to a variable. This is called \textbf{tidy
data}.\footnote{For more information on tidy data see the
  \href{https://simplystatistics.org/2016/02/17/non-tidy-data/}{blog}
  and the \href{https://r4ds.had.co.nz/tidy-data.html\#pivoting}{book}.}
The data frame for the stroke study introduced in the previous lesson
had patients as the cases and there were three variables recorded for
each patient. If we are thinking of patients as the unit of observation,
then this data is tidy.

\begin{verbatim}
## # A tibble: 10 x 3
##    group   outcome30 outcome365
##    <chr>   <chr>     <chr>     
##  1 control no_event  no_event  
##  2 trmt    no_event  no_event  
##  3 control no_event  no_event  
##  4 trmt    no_event  no_event  
##  5 trmt    no_event  no_event  
##  6 control no_event  no_event  
##  7 trmt    no_event  no_event  
##  8 control no_event  no_event  
##  9 control no_event  no_event  
## 10 control no_event  no_event
\end{verbatim}

If we think of an outcome as a unit of observation then it is not tidy
since the two outcome columns are variable values, time. The tidy data
for this case would be:

\begin{verbatim}
## # A tibble: 10 x 4
##    patient_id group   time  result  
##         <int> <chr>   <chr> <chr>   
##  1          1 control month no_event
##  2          1 control year  no_event
##  3          2 trmt    month no_event
##  4          2 trmt    year  no_event
##  5          3 control month no_event
##  6          3 control year  no_event
##  7          4 trmt    month no_event
##  8          4 trmt    year  no_event
##  9          5 trmt    month no_event
## 10          5 trmt    year  no_event
\end{verbatim}

There are three interrelated rules which make a dataset tidy:

\begin{enumerate}
\def\labelenumi{\arabic{enumi}.}
\tightlist
\item
  Each variable must have its own column.\\
\item
  Each observation must have its own row.\\
\item
  Each value must have its own cell.
\end{enumerate}

Why ensure that your data is tidy? There are two main advantages:

\begin{enumerate}
\def\labelenumi{\arabic{enumi}.}
\item
  There's a general advantage to picking one consistent way of storing
  data. If you have a consistent data structure, it's easier to learn
  the tools that work with it because they have an underlying
  uniformity.
\item
  There's a specific advantage to placing variables in columns because
  it allows \texttt{R}'s vectorised nature to shine. This will be more
  clear as the semester progresses. Since most built-in \texttt{R}
  functions work with vectors of values, it makes transforming tidy data
  feel particularly natural.
\end{enumerate}

Data frames are a convenient way to record and store data. If another
individual or case is added to the data set, an additional row can be
easily added. Similarly, another column can be added for a new variable.

\begin{quote}
Exercise\\
We consider a publicly available data set that summarizes information
about the 3,143 counties in the United States, and we call this the
\texttt{county} data set. This data set includes information about each
county: its name, the state where it resides, its population in 2000 and
2010, per capita federal spending, poverty rate, and four additional
characteristics. This data set is part of \texttt{openintro} and is
called \texttt{county}. How might these data be organized in a data
matrix? \footnote{Each county may be viewed as a case, and there are ten
  pieces of information recorded for each case. A table with 3,143 rows
  and 10 columns could hold these data, where each row represents a
  county and each column represents a particular piece of information.}
\end{quote}

Using \texttt{R}, we will display seven rows of the \texttt{county} data
frame. The variables are summarized in help menu built into the
\texttt{openintro} package.\footnote{\href{http://quickfacts.census.gov/qfd/index.html}{These
  data were collected from the US Census website.}}

\begin{Shaded}
\begin{Highlighting}[]
\KeywordTok{head}\NormalTok{(county,}\DataTypeTok{n=}\DecValTok{7}\NormalTok{)}
\end{Highlighting}
\end{Shaded}

\begin{verbatim}
##             name   state pop2000 pop2010 fed_spend poverty homeownership
## 1 Autauga County Alabama   43671   54571  6.068095    10.6          77.5
## 2 Baldwin County Alabama  140415  182265  6.139862    12.2          76.7
## 3 Barbour County Alabama   29038   27457  8.752158    25.0          68.0
## 4    Bibb County Alabama   20826   22915  7.122016    12.6          82.9
## 5  Blount County Alabama   51024   57322  5.130910    13.4          82.0
## 6 Bullock County Alabama   11714   10914  9.973062    25.3          76.9
## 7  Butler County Alabama   21399   20947  9.311835    25.0          69.0
##   multiunit income med_income
## 1       7.2  24568      53255
## 2      22.6  26469      50147
## 3      11.1  15875      33219
## 4       6.6  19918      41770
## 5       3.7  21070      45549
## 6       9.9  20289      31602
## 7      13.7  16916      30659
\end{verbatim}

\hypertarget{types-of-variables}{%
\subsubsection{Types of variables}\label{types-of-variables}}

Examine the \texttt{fed\_spend}, \texttt{pop2010}, and \texttt{state}
variables in the \texttt{county} data set. Each of these variables is
inherently different from the other three yet many of them share certain
characteristics.

First consider \texttt{fed\_spend}, which is said to be a
\textbf{numerical variable} since it can take a wide range of numerical
values, and it is sensible to add, subtract, or take averages with those
values. On the other hand, we would not classify a variable reporting
telephone area codes as numerical; even though area codes are made up of
numerical digits, their average, sum, and difference have no clear
meaning.

The \texttt{pop2010} variable is also numerical; it is sensible to add,
subtract, or take averages with those values, although it seems to be a
little different than \texttt{fed\_spend}. This variable of the
population count can only be a whole non-negative number (\(0\), \(1\),
\(2\), \(...\)). For this reason, the population variable is said to be
\textbf{discrete} since it can only take specific numerical values. On
the other hand, the federal spending variable is said to be
\textbf{continuous}. Now technically, there are no truly continuous
numerical variables since all measurements are finite up to some level
of accuracy or measurement precision. However, in this course we will
treat both variables types of numerical variables the same, that is as
continuous variables. The only place this will be different is in
probability models which we see in the next block.

The variable \textbf{state} can take up to 51 values after accounting
for Washington, DC: \emph{AL}, \ldots{}, and \emph{WY}. Because the
responses themselves are categories, \texttt{state} is called a
\textbf{categorical} variable,\footnote{Sometimes also called a
  \textbf{nominal} variable.} and the possible values are called the
variable's \textbf{levels}.

\begin{figure}
\centering
\includegraphics{02-Data-Basics-Notes_files/figure-latex/unnamed-chunk-7-1.pdf}
\caption{Taxonomy of Variables}
\end{figure}

Finally, consider a hypothetical variable on education, which describes
the highest level of education completed and takes on one of the values
\emph{noHS}, \emph{HS}, \emph{College} or \emph{Graduate\_school}. This
variable seems to be a hybrid: it is a categorical variable but the
levels have a natural ordering. A variable with these properties is
called an \textbf{ordinal} variable. To simplify analyses, any ordinal
variables in this course will be treated as categorical variables. In
\texttt{R} categorical variables can be treated in different ways; one
of the key differences is that we can leave them as character values or
as factors. When \texttt{R} handles factors, it is only concerned about
the \emph{levels} of values of the factors. We will learn more about
this as the semester progresses.

Figure 1 captures this classification of variables.

\begin{quote}
Exercise\\
Data were collected about students in a statistics course. Three
variables were recorded for each student: number of siblings, student
height, and whether the student had previously taken a statistics
course. Classify each of the variables as continuous numerical, discrete
numerical, or categorical.
\end{quote}

The number of siblings and student height represent numerical variables.
Because the number of siblings is a count, it is discrete. Height varies
continuously, so it is a continuous numerical variable. The last
variable classifies students into two categories -- those who have and
those who have not taken a statistics course -- which makes this
variable categorical.

\begin{quote}
Exercise\\
Consider the variables \texttt{group} and \texttt{outcome30} from the
stent study in the case study lesson. Are these numerical or categorical
variables? \footnote{There are only two possible values for each
  variable, and in both cases they describe categories. Thus, each is a
  categorical variable.}
\end{quote}

\hypertarget{relationships-between-variables}{%
\subsubsection{Relationships between
variables}\label{relationships-between-variables}}

Many analyses are motivated by a researcher looking for a relationship
between two or more variables, this is the heart of statistical
modeling. A social scientist may like to answer some of the following
questions:

\begin{enumerate}
\def\labelenumi{\arabic{enumi}.}
\tightlist
\item
  Is federal spending, on average, higher or lower in counties with high
  rates of poverty?\\
\item
  If homeownership is lower than the national average in one county,
  will the percent of multi-unit structures in that county likely be
  above or below the national average?
\end{enumerate}

To answer these questions, data must be collected, such as the
\texttt{county} data set. Examining summary statistics could provide
insights for each of the three questions about counties. Additionally,
graphs can be used to visually summarize data and are useful for
answering such questions as well.

Scatterplots are one type of graph used to study the relationship
between two numerical variables. Figure 2 compares the variables
\texttt{fed\_spend} and \texttt{poverty}. Each point on the plot
represents a single county. For instance, the highlighted dot
corresponds to County 1088 in the \texttt{county} data set: Owsley
County, Kentucky, which had a poverty rate of 41.5\% and federal
spending of \$21.50 per capita. The dense cloud in the scatterplot
suggests a relationship between the two variables: counties with a high
poverty rate also tend to have slightly more federal spending. We might
brainstorm as to why this relationship exists and investigate each idea
to determine which is the most reasonable explanation.

\begin{figure}
\centering
\includegraphics{02-Data-Basics-Notes_files/figure-latex/unnamed-chunk-8-1.pdf}
\caption{A scatterplot showing fed\_spend against poverty. Owsley County
of Kentucky, with a poverty rate of 41.5\% and federal spending of
\$21.50 per capita, is highlighted.}
\end{figure}

\begin{quote}
Exercise\\
Examine the variables in the \texttt{email50} data set. Create two
questions about the relationships between these variables that are of
interest to you.\footnote{Two sample questions: (1) Intuition suggests
  that if there are many line breaks in an email then there would also
  tend to be many characters: does this hold true? (2) Is there a
  connection between whether an email format is plain text (versus HTML)
  and whether it is a spam message?}
\end{quote}

The \texttt{fed\_spend} and \texttt{poverty} variables are said to be
associated because the plot shows a discernible pattern. When two
variables show some connection with one another, they are called
\textbf{associated variables}. Associated variables can also be called
\textbf{dependent} variables and vice-versa.

\begin{quote}
Example\\
The relationship between the homeownership rate and the percent of units
in multi-unit structures (e.g.~apartments, condos) is visualized using a
scatterplot in Figure 3. Are these variables associated?
\end{quote}

It appears that the larger the fraction of units in multi-unit
structures, the lower the homeownership rate. Since there is some
relationship between the variables, they are associated.

\begin{figure}
\centering
\includegraphics{02-Data-Basics-Notes_files/figure-latex/unnamed-chunk-9-1.pdf}
\caption{A scatterplot of the homeownership rate versus the percent of
units that are in multi-unit structures for all 3,143 counties.}
\end{figure}

Because there is a downward trend in Figure 3 -- counties with more
units in multi-unit structures are associated with lower homeownership
-- these variables are said to be \textbf{negatively associated}. A
\textbf{positive association} is shown in the relationship between the
\texttt{poverty} and \texttt{fed\_spend} variables represented in Figure
2, where counties with higher poverty rates tend to receive more federal
spending per capita.

If two variables are not associated, then they are said to be
\textbf{independent}. That is, two variables are independent if there is
no evident relationship between the two.

\begin{quote}
A pair of variables are either related in some way (associated) or not
(independent). No pair of variables is both associated and independent.
\end{quote}

\hypertarget{creating-a-scatterplot}{%
\subsubsection{Creating a scatterplot}\label{creating-a-scatterplot}}

In this section we will create a simple scatterplot and then ask you to
create one on your own. First we will recreate the scatterplot seen in
Figure 2. This figure uses the \texttt{county} data set.

Here are two questions:

\emph{What do we want \texttt{R} to do?} and

\emph{What must we give \texttt{R} for it to do this?}

We want \texttt{R} to create a scatterplot and to do this it needs, at a
minimum, the data object, what we want on the \(x\)-axis, and what we
want on the \(y\)-axis. More information on
\href{https://cran.r-project.org/web/packages/ggformula/vignettes/ggformula-blog.html}{\texttt{ggformula}}
can be found by clicking on the link.\footnote{\url{https://cran.r-project.org/web/packages/ggformula/vignettes/ggformula-blog.html}}

\begin{Shaded}
\begin{Highlighting}[]
\NormalTok{county }\OperatorTok
\StringTok{  }\KeywordTok{gf_point}\NormalTok{(fed_spend}\OperatorTok{~}\NormalTok{poverty)}
\end{Highlighting}
\end{Shaded}

\includegraphics{02-Data-Basics-Notes_files/figure-latex/unnamed-chunk-10-1.pdf}

This plot is bad, there are poor axis labels, no title, dense clustering
of points, the \(y\)-axis is being driven by a couple of extreme points.
We will need to clear this up. Again, try to read the code and use
\texttt{help()} or \texttt{?} to determine the purpose of each command.

\begin{Shaded}
\begin{Highlighting}[]
\NormalTok{county }\OperatorTok
\StringTok{  }\KeywordTok{filter}\NormalTok{(fed_spend}\OperatorTok{<}\DecValTok{32}\NormalTok{) }\OperatorTok
\StringTok{  }\KeywordTok{gf_point}\NormalTok{(fed_spend}\OperatorTok{~}\NormalTok{poverty,}
           \DataTypeTok{xlab=}\StringTok{"Poverty Rate (Percent)"}\NormalTok{, }
           \DataTypeTok{ylab=}\StringTok{"Federal Spending Per Capita"}\NormalTok{,}
           \DataTypeTok{title=}\StringTok{"A scatterplot showing fed_spend against poverty"}\NormalTok{, }
           \DataTypeTok{subtitle =}  \StringTok{"Owsley County of Kentucky"}\NormalTok{,}
           \DataTypeTok{cex=}\DecValTok{1}\NormalTok{,}\DataTypeTok{alpha=}\FloatTok{0.2}\NormalTok{) }\OperatorTok
\StringTok{  }\KeywordTok{gf_theme}\NormalTok{(}\KeywordTok{theme_classic}\NormalTok{())}
\end{Highlighting}
\end{Shaded}

\includegraphics{02-Data-Basics-Notes_files/figure-latex/unnamed-chunk-11-1.pdf}

\begin{quote}
Exercise\\
Create the scatterplot in Figure 3.
\end{quote}

\hypertarget{file-creation-information}{%
\subsubsection{File creation
information}\label{file-creation-information}}

\begin{itemize}
\tightlist
\item
  File creation date: 2020-05-13
\item
  Windows version: Windows 10 x64 (build 15063)
\item
  R version 3.6.3 (2020-02-29)
\item
  \texttt{mosaic} package version: 1.6.0
\item
  \texttt{tidyverse} package version: 1.3.0
\item
  \texttt{openintro} package version": 1.7.1
\end{itemize}

\end{document}
